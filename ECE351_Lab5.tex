%%%%%%%%%%%%%%%%%%%%%%%%%%%%%%%%%%%%%%%%%%%
%%% DOCUMENT PREAMBLE %%%
\documentclass[12pt]{report}
\usepackage[english]{babel}
%\usepackage{natbib}
\usepackage{url}
\usepackage[utf8x]{inputenc}
\usepackage{amsmath}
\usepackage{graphicx}
\graphicspath{{images/}}
\usepackage{parskip}
\usepackage{fancyhdr}
\usepackage{vmargin}
\usepackage{listings}
\usepackage{hyperref}
\usepackage{xcolor}
\definecolor{codegreen}{rgb}{0,0.6,0}
\definecolor{codegray}{rgb}{0.5,0.5,0.5}
\definecolor{codeblue}{rgb}{0,0,0.95}
\definecolor{backcolour}{rgb}{0.95,0.95,0.92}
\lstdefinestyle{mystyle}{
backgroundcolor=\color{backcolour},
commentstyle=\color{codegreen},
keywordstyle=\color{codeblue},
numberstyle=\tiny\color{codegray},
stringstyle=\color{codegreen},
basicstyle=\ttfamily\footnotesize,
breakatwhitespace=false,
breaklines=true,
captionpos=b,
keepspaces=true,
numbers=left,
numbersep=5pt,
showspaces=false,
showstringspaces=false,
showtabs=false,
tabsize=2
}
\lstset{style=mystyle}
\setmarginsrb{3 cm}{2.5 cm}{3 cm}{2.5 cm}{1 cm}{1.5 cm}{1 cm}{1.5 cm}
\title{System Step Respone}
% Title
\author{Zachary Pfaff}
% Author
\date{February 15th, 2022}
% Date
\makeatletter
\let\thetitle\@title
\let\theauthor\@author
\let\thedate\@date
\makeatother
\pagestyle{fancy}
\fancyhf{}
\rhead{\theauthor}
\lhead{\thetitle}
\cfoot{\thepage}
%%%%%%%%%%%%%%%%%%%%%%%%%%%%%%%%%%%%%%%%%%%%
\begin{document}
%%%%%%%%%%%%%%%%%%%%%%%%%%%%%%%%%%%%%%%%%%%%%%%%%%%%%%%%%%%%%%%%%%%%%%%%%%
%%%%%%%%%%%%%%%
\begin{titlepage}
\centering
\vspace*{0.5 cm}
% \includegraphics[scale = 0.075]{bsulogo.png}\\[1.0 cm] % 
\includegraphics[width = 5 cm]{University_of_Idaho_seal.svg.png}\\
\begin{center}    \textsc{\Large   ECE 351}\\[2.0 cm]
\end{center}% University Name
\textsc{\Large Signals and Systems  }\\[0.5 cm] % Course 
Lab 4
\rule{\linewidth}{0.2 mm} \\[0.4 cm]
{ \huge \bfseries \thetitle}\\
\rule{\linewidth}{0.2 mm} \\[1.5 cm]
\begin{minipage}{0.4\textwidth}
\begin{flushleft} \large
%\emph{Submitted To:}\\
% Name\\
% Affiliation\\
%contact info\\
\end{flushleft}
\end{minipage}~
\begin{minipage}{0.4\textwidth}
\begin{flushright} \large
\emph{Submitted By :} \\
Zachary Pfaff\\https://github.com/ZachPfaff
\end{flushright}
\end{minipage}\\[2 cm]
% \includegraphics[scale = 0.5]{PICMathLogo.png}
\end{titlepage}
%%%%%%%%%%%%%%%%%%%%%%%%%%%%%%%%%%%%%%%%%%%%%%%%%%%%%%%%%%%%%%%%%%%%%%%%%%
%%%%%%%%%%%%%%%
\tableofcontents
\pagebreak
%%%%%%%%%%%%%%%%%%%%%%%%%%%%%%%%%%%%%%%%%%%%%%%%%%%%%%%%%%%%%%%%%%%%%%%%%%
%%%%%%%%%%%%%%%
\renewcommand{\thesection}{\arabic{section}}
\setlength{\parindent}{20pt}

\maketitle
\section{Introduction}
\hspace{\parindent}In the last few labs, signals were evaluated using a user-defined convolution function. Now that Laplace transforms have been introduced in ECE 350, the following lab will focus on evaluating the impulse response of a function using Laplace transforms. The goal of lab 5 is to plot the impulse response and step response of an RLC circuit using hand calculations and built in python tools. \par

\section{Equations}
\hspace{\parindent}Before starting any python code, we had to evaluate an RLC circuit by using Laplace transforms to find the time domain step and impulse responses. The RLC circuit is viewed below along with its respected resistor, inductor, and capacitor values. \par
\begin{figure}[h!]
    \centering
    \includegraphics[width = 12 cm]{RLC Filter.png}
    \caption{RLC Filter where R=1kΩ, L = 27 mH, C = 100 nF}
    \label{Figure x:}
\end{figure}
\hspace{\parindent}To start this problem, the transfer function for Vout/Vin needed to be found. The equation below shows the simplified transfer function which was given help from the lab TA. \par
\[H(s) = \frac{\frac{1}{RC}s}{s^2+\frac{1}{RC}s + (\frac{1}{\sqrt{LC}})^2}\]
\hspace{\parindent}Next, the impulse response was to be evaluated. The equations for the impulse are viewed below. \par
\[p = a +jw = \frac{-1}{2RC} ± \frac{1}{2}\sqrt{(\frac{1}{RC})^2 - 4(\frac{1}{LC})}\]
\[|g| = \sqrt{[\frac{-1}{2}(\frac{1}{RC}^2)]^2+[\frac{1}{2RC}\sqrt{(\frac{1}{RC})^2-4(\frac{1}{LC})}]^2}\]
\[∠g = tan^{-1}(\frac{\frac{1}{2RC}\sqrt{(\frac{1}{RC})^2-4(\frac{1}{LC})}}{\frac{-1}{2}(\frac{1}{RC})^2})\]
The final impulse response was determined using the sin method as seen below.
\[h(t) = \frac{|g|}{w}e^{at}sin(wt + ∠g)u(t)\]
\section{Methodology}
\hspace{\parindent}The first part of lab 5 calls for a comparison between the impulse response using the hand calculated method vs the scipy.signal.impulse() function. The code I used to implement the hand calculated impulse response can be seen below in listing 1. Notice that all the variables involved in the impulse response equation above were already calculated out and set into the individual variables that make up the equation. \par
\begin{lstlisting}[language=Python, caption=Hand Calculated Function Code]
#Variable declarations
R = 1000
L = 0.027
C = 0.0000001
w = 18581
a = -5000
mg = 192455904
pg = 1.83

#hand calculated function
y = ((mg/w)*np.exp(a*t)*np.sin(w * t + pg)) * step(t)
\end{lstlisting}
\hspace{\parindent}The next listing demonstrates the same impulse response, but this time using the scipy.signal.impulse() function. I used the code from the Lab 5 handout as reference for the impulse() function syntax. \par
\begin{lstlisting}[language=Python, caption=Impulse() Function Code]
steps = 0.00001
t = np.arange(0, 1.2e-3 + steps, steps)

num = [0, 10000, 0]
den = [1, 10000, 370370370]

#impulse response
tout, yout = sig.impulse((num, den), T = t)
\end{lstlisting}
\hspace{\parindent}Part 2 of the lab involved finding the step response of the function for the final value theorem to be performed. To do this, I used scipy.signal.step() to plot the step response. The code for the step response used the same numerator and denominator values as the impulse response function, so it looks similar to listing 2. Listing 3 below shows the code used to implement the step response of the function. \par
\begin{lstlisting}[language=Python, caption=Step() Response Code]
steps = 0.00001
t = np.arange(0, 1.2e-3 + steps, steps)

num = [0, 10000, 0]
den = [1, 10000, 370370370]

#step response
xout, sout = sig.step((num, den), T = t)
\end{lstlisting}
\hspace{\parindent}Following this code, the final value theorem for the step response was to be performed to determine if the output is similar to the hand calulated impulse response and the impulse() funcion impulse response. The results for my final value theorem can be seen below in the following equations. \par
\[\lim_{t\to\infty}{f(t)} = \lim_{s\to0}[sF(s)]\]
\[\lim_{s\to0}[sF(s)] = \lim_{s\to0}[s\frac{\frac{1}{RC}s}{s^2+\frac{1}{RC}s + (\frac{1}{\sqrt{LC}})^2}] = 0\]
\hspace{\parindent}So the step response of the function using the final value theorem is equal to zero. When the step response is plotted, it should start at zero and balance out to remain at zero. \par

\section{Results}
\hspace{\parindent}With code for the impulse response implemented, I graphed both methods side by side to compare the results. The following figures below shows the impulse response of the time domain function using hand calculation compared to the impulse() function. As viewed below, both methods output the same exact graph. \par
\begin{figure}[h!]
    \centering
    \includegraphics[width = 13 cm]{Hand Calculated.png}
    \caption{Hand calculated method}
    \label{Figure x:}
\end{figure}
\begin{figure}[h!]
    \centering
    \includegraphics[width = 13 cm]{Impulse().png}
    \caption{Impulse() method}
    \label{Figure x:}
\end{figure}
\hspace{\parindent}The next graph to be plotted is the step response function using the scipy.signal.step() function. The plot for this function can be viewed below in figure 4. Notice that the plot starts at zero and resembles a similar shape to the impulse response except shifted to the right. This makes sense because the voltage output of the circuit will be equal to zero at any point before an input voltage is supplied, this idea will be further expanded upon in the questions section of this report. \par
\begin{figure}[h!]
    \centering
    \includegraphics[width = 13 cm]{Step Response.png}
    \caption{Step Response of F(s)}
    \label{Figure x:}
\end{figure}
\section{Conclusion}
\hspace{\parindent}Implementation of lab 5 went relatively smoothly, I ended up having some trouble with my initial calculations of the variables that went into the hand calculated function. It took a few attempts to solve the variables by hand to match the plot with the impulse() function plot, spending the time to do so however helped me in my understanding of the sin method. My implementation for part 2 of lab 5 also went well since it was similar in syntax to the impulse() function. Doing so also forced me to take a closer look at the final value theorem and where to use it when applicable.

\section{Questions}
1. Explain the result of the Final Value Theorem from Part 2 Task 2 in terms of the physical circuit components\\\\
Before the circuit is energized, the inductor will act as a short and the capacitor will act open since no energy is flowing through the two components. When this is analyzed, Vout has no voltage which means at time zero, the output voltage must also be zero. This makes sense graphically because the step response starts at zero.\\\\
2. Leave any feedback on the clarity of the expectations, instructions, and deliverables.\\\\
I did not have any problems (I believe) following the lab handout details.
\end{document}
